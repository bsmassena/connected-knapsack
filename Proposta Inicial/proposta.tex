\documentclass[14pt]{extarticle}
\usepackage[utf8]{inputenc}
\usepackage{amsmath}
\usepackage{indentfirst}

\setlength{\parindent}{4em}
\setlength{\parskip}{1em}

\title{Proposta Inicial Para o Trabalho Final da Disciplina de Otimização Combinatória}
\author{Bruno Santana Massena de Lima }
\date{Novembro 2018}

\usepackage{natbib}
\usepackage{graphicx}

\begin{document}

\maketitle

\section{Problema da Mochila Conexa}
\noindent
Formulação:\\
\begin{gather*}
\begin{aligned}
\textup{max}\qquad \sum_{i\in[V]} v_i c_i\\
\textup{s.t.}\qquad \sum_{i \in [V]} c_i w_i     &\leq U\\
                    f_{ij}                      &\leq M E_{ij}, & \forall i,j \in [V]\\
                    f_{ij}                      &\leq M c_i,    & \forall i,j \in [V]\\
                    f_{ij}                      &\leq M c_j,    & \forall i,j \in [V]\\
                    \sum_{j \in [V]} f_{ij} + 1 &\leq \sum_{j \in [V]} f_{ji} + M(s_i + (1 - c_i)), & \forall i \in [V]\\
                    \sum_{j \in [V]} f_{ij} + 1 &\geq \sum_{j \in [V]} f_{ji} - M(s_i + (1 - c_i)), & \forall i \in [V]\\
                    \sum_{j \in [V]} f_{ij}     &\geq \sum_{i \in [V]} c_i - 1 - M(1 - s_i), & \forall i \in [V]\\
                    \sum_{i \in [V]} f_{ij}     &\leq M(1 - s_j), & \forall j \in [V]\\
                    \sum_{i \in [V]} s_i        &= 1\\
                    0 &\leq c_i \leq 1, & \forall i \in [V]\\
                    0 &\leq s_i \leq 1, & \forall i \in [V]\\
                    f_{ij}                      &\geq 0, & \forall i,j \in [V]\\
\end{aligned}
\end{gather*}
\\
\\
\\
\noindent
Nesta formulação:\\
\hspace*{2ex} E representa a matriz de adjacência do grafo de entrada\\
\hspace*{2ex} V representa a quantidade de vértices do grafo de entrada\\
\hspace*{2ex} U representa a capacidade máxima da mochila\\
\hspace*{2ex} $v_i$ representa o valor do vértice \textit{i}\\
\hspace*{2ex} $w_i$ representa o peso do vértice \textit{i}\\
\hspace*{2ex} $c_i$ indica se o vértice \textit{i} foi escolhido (1) ou não (0)\\
\hspace*{2ex} $s_i$ indica se o vértice \textit{i} é o vértice de origem do fluxo (1) ou não (0)\\
\hspace*{2ex} $f_{ij}$ representa o fluxo do vértice \textit{i} ao vértice \textit{j}\\

\section{Elementos do Simulated Annealing}

A \textbf{representação de uma solução} se dará por uma lista de inteiros, cada um representando um item/vértice escolhido pertencente à instância do problema.

A \textbf{solução inicial} será feita de forma aleatória. O algortimo irá selecionar um vértice inicial com peso menor que a capacidade máxima da mochila. A partir disso, o algoritmo irá encontrar vértices adjacentes aos já adicionados na solução e adicioná-los à mesma. Este processo se repetirá até que não seja possível adicionar mais vértices à solução sem que o peso total dos vértices ultrapasse a capacidade máxima da mochila. Além disso, também penso em implementar outros métodos de escolha de solução inicial, algo como uma escolha gulosa, ou \textit{best-fit/worst-fit}

\textbf{Vizinhanças:} Inicialmente, a vizinhança será aleatória. Será selecionado um item da solução atual para ser removido e um item para ser adicionado. Caso a solução gerada seja factível, ela será retornada. Caso contrário, o precesso será repetido. Para diversificar a quantidade de itens na mochila, também pode ocorrer de algum dos vértices não ser escolhido, ou seja, um item será adicionado sem a necessidade de remover outro, ou vice-versa.
Além desta, também pretendo implementar outros tipos de vizinhanças, e discutir qual obteve melhores resultados no relatório final.

\textbf{Temperaturas:} Para definir as temperaturas inicial e final, usarei os métodos descritos nas notas de aula:

''Exemplo 10.2 (Temperatura inicial)
Define uma probabilidade pi
Executa uma versão rápida (I pequeno) do algoritmo
para determinar uma temperatura inicial tal que um movimento é aceito
com probabilidade pi

Exemplo 10.3 (Temperatura final)
Define uma probabilidade pf
Para cada nível de temperatura em que os movimentos
foram aceitos com probabilidade menos que pf
incrementa um contador.
Zera o contador caso uma nova melhor solução é encontrada. Caso o contador
chega em 5, termina.''


\end{document}
